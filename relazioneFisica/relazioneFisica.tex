% !TeX spellcheck = it_IT 
\documentclass[11pt]{article}
\usepackage[italian]{babel}
\usepackage[T1]{fontenc}
\usepackage[utf8]{inputenc}
\usepackage{graphicx}
\usepackage{todonotes}
\usepackage{listings}
\usepackage{caption}
\usepackage{subcaption}
\usepackage{abbrevs}



\definecolor{dkgreen}{rgb}{0,0.6,0}
\definecolor{gray}{rgb}{0.5,0.5,0.5}
\definecolor{mauve}{rgb}{0.58,0,0.82}

\lstset{
  frame=single,
  captionpos=b,
  language=Java,
  aboveskip=3mm,
  belowskip=3mm,
  showstringspaces=false,
  columns=flexible,
  basicstyle={\small\ttfamily},
  numbers=none,
  numberstyle=\tiny\color{gray},
  keywordstyle=\color{blue},
  commentstyle=\color{dkgreen},
  stringstyle=\color{mauve},
  breaklines=true,
  breakatwhitespace=true,
  tabsize=3
}

\makeatletter
\def\cleardoublepage{
	\clearpage\if@twoside \ifodd\c@page\else
	\hbox{}
	\thispagestyle{empty}
	\newpage
	\if@twocolumn\hbox{}\newpage\fi\fi\fi
}

\makeatother

\setlength{\textwidth}{14cm}
\setlength{\textheight}{21cm}
\setlength{\footskip}{3cm}

\setlength{\hoffset}{0pt}
\setlength{\voffset}{0pt}

\setlength{\oddsidemargin}{1cm}
\setlength{\evensidemargin}{1cm}

\begin{document}
	\title{Report progetto fisica \\dei sistemi complessi\large}
	
	\author{Alessandro Cordella\large \\Natale Vadalà\\alessandro.cordella@studio.unibo.it\\natale.vadala@studio.unibo.it}
	\date{Novembre 2018}
	\maketitle
	\newpage
	\tableofcontents
	\newpage
\section{Introduzione}
L'obiettivo del presente lavoro è stato quello di costruire un robot capace di muoversi su ruote. Il dispositivo ottenuto è controllabile da un'interfaccia web ed è possibile visionare ciò che è posto di fronte al robot tramite una webcam.\\Nelle seguenti sezioni verranno descritti i componenti hardware utilizzati, le tecniche software utilizzate e verranno presentati eventuali sviluppi futuri.
\section{Hardware} 
\section{Software}
Per garantire il controllo del dispositivo tramite un'interfaccia web è stato necessario progettare un server al quale un client può collegarsi per ottenere il controllo del dispositivo. È stato utilizzato il framework javascript \textit{NodeJs} per gestire sia la parte client che server dell'applicazione. Nelle seguenti sezioni verrà descritta la struttura gerarchica delle cartelle e le funzionalità presenti nei file al loro interno.
\subsection{Server}
I file presenti all'interno della cartella \textit{server} contengono le funzioni per gestire la parte server del sistema, mentre all'interno delle sue sottocartelle sono presenti altri file che verranno descritti in seguito.\\
\subsubsection{main.js}
Il file \textit{main.js} è il file principale del server. Al suo interno sono presenti le funzioni di inizializzazione del sistema e per la gestione dello streaming video.\\
Per avviare il server è necessario digitare il seguente comando:
\begin{lstinputlisting}[caption={Avvio server},basicstyle=\tiny]{codice/startServer}\end{lstinputlisting}
\begin{itemize}
	\item \textbf{node} nodejs;
	\item \textbf{main.js} file del server;
	\item \textbf{ciao} parola chiave; \todo{cambiare ciao}
	\item \textbf{8081 8082 8083}  qualcosa \todo{scrivere}
\end{itemize}
Una volta lanciato il comando, sarà possibile accedere all'interfaccia web digitando l'indirizzo IP del raspberry seguito da \textit{:8083}.
\begin{lstinputlisting}[caption={Esempio indirizzo server},basicstyle=\tiny]{codice/indirizzo}\end{lstinputlisting}
\subsubsection{routes.js}
Questo file è quello che gestisce la comunicazione client server, associando ad ogni interazione del client sulla pagina web una specifica azione del server. Gestisce quindi le funzioni che regolano la webcam e il movimento. In particolare viene utilizzata una libreria di nodejs per eseguire la libreria Python \textit{AMSpi} per la gestione dei motori.
\subsubsection{AMSpi}
\textit{AMSpi} è la libreria Python utilizzata per controllare i motori. Al suo interno è presente il file \textit{movements.py}. È il file che usa le funzioni della libreria per controllare i motori.
\subsection{Client}
I file presenti all'interno della cartella \textit{public} sono quelli che gestiscono la parte client dell'applicazione.\\
\subsubsection{ex3-1.html} 
\todo{immagine}
È il file contenente la struttura del sito web.
\subsubsection{css}
All'interno di questa cartella sono presenti i file che regolano lo stile della pagina.
\subsubsection{js}
In questa cartella sono contenuti i file che regolano le interazioni dell'utente con la pagina e comunicano al server le azioni intraprese da esso.

\section{Ulteriori sviluppi e Conclusioni}	
%\newpage	
%\bibliographystyle{abbrv}
%\bibliography{Bib}
\end{document}